% Options for packages loaded elsewhere
% Options for packages loaded elsewhere
\PassOptionsToPackage{unicode}{hyperref}
\PassOptionsToPackage{hyphens}{url}
\PassOptionsToPackage{dvipsnames,svgnames,x11names}{xcolor}
%
\documentclass[
  letterpaper,
  DIV=11,
  numbers=noendperiod]{scrartcl}
\usepackage{xcolor}
\usepackage{amsmath,amssymb}
\setcounter{secnumdepth}{-\maxdimen} % remove section numbering
\usepackage{iftex}
\ifPDFTeX
  \usepackage[T1]{fontenc}
  \usepackage[utf8]{inputenc}
  \usepackage{textcomp} % provide euro and other symbols
\else % if luatex or xetex
  \usepackage{unicode-math} % this also loads fontspec
  \defaultfontfeatures{Scale=MatchLowercase}
  \defaultfontfeatures[\rmfamily]{Ligatures=TeX,Scale=1}
\fi
\usepackage{lmodern}
\ifPDFTeX\else
  % xetex/luatex font selection
\fi
% Use upquote if available, for straight quotes in verbatim environments
\IfFileExists{upquote.sty}{\usepackage{upquote}}{}
\IfFileExists{microtype.sty}{% use microtype if available
  \usepackage[]{microtype}
  \UseMicrotypeSet[protrusion]{basicmath} % disable protrusion for tt fonts
}{}
\makeatletter
\@ifundefined{KOMAClassName}{% if non-KOMA class
  \IfFileExists{parskip.sty}{%
    \usepackage{parskip}
  }{% else
    \setlength{\parindent}{0pt}
    \setlength{\parskip}{6pt plus 2pt minus 1pt}}
}{% if KOMA class
  \KOMAoptions{parskip=half}}
\makeatother
% Make \paragraph and \subparagraph free-standing
\makeatletter
\ifx\paragraph\undefined\else
  \let\oldparagraph\paragraph
  \renewcommand{\paragraph}{
    \@ifstar
      \xxxParagraphStar
      \xxxParagraphNoStar
  }
  \newcommand{\xxxParagraphStar}[1]{\oldparagraph*{#1}\mbox{}}
  \newcommand{\xxxParagraphNoStar}[1]{\oldparagraph{#1}\mbox{}}
\fi
\ifx\subparagraph\undefined\else
  \let\oldsubparagraph\subparagraph
  \renewcommand{\subparagraph}{
    \@ifstar
      \xxxSubParagraphStar
      \xxxSubParagraphNoStar
  }
  \newcommand{\xxxSubParagraphStar}[1]{\oldsubparagraph*{#1}\mbox{}}
  \newcommand{\xxxSubParagraphNoStar}[1]{\oldsubparagraph{#1}\mbox{}}
\fi
\makeatother


\usepackage{longtable,booktabs,array}
\usepackage{calc} % for calculating minipage widths
% Correct order of tables after \paragraph or \subparagraph
\usepackage{etoolbox}
\makeatletter
\patchcmd\longtable{\par}{\if@noskipsec\mbox{}\fi\par}{}{}
\makeatother
% Allow footnotes in longtable head/foot
\IfFileExists{footnotehyper.sty}{\usepackage{footnotehyper}}{\usepackage{footnote}}
\makesavenoteenv{longtable}
\usepackage{graphicx}
\makeatletter
\newsavebox\pandoc@box
\newcommand*\pandocbounded[1]{% scales image to fit in text height/width
  \sbox\pandoc@box{#1}%
  \Gscale@div\@tempa{\textheight}{\dimexpr\ht\pandoc@box+\dp\pandoc@box\relax}%
  \Gscale@div\@tempb{\linewidth}{\wd\pandoc@box}%
  \ifdim\@tempb\p@<\@tempa\p@\let\@tempa\@tempb\fi% select the smaller of both
  \ifdim\@tempa\p@<\p@\scalebox{\@tempa}{\usebox\pandoc@box}%
  \else\usebox{\pandoc@box}%
  \fi%
}
% Set default figure placement to htbp
\def\fps@figure{htbp}
\makeatother





\setlength{\emergencystretch}{3em} % prevent overfull lines

\providecommand{\tightlist}{%
  \setlength{\itemsep}{0pt}\setlength{\parskip}{0pt}}



 


% load packages
\usepackage{geometry}
\usepackage{xcolor}
\usepackage{eso-pic}
\usepackage{fancyhdr}
\usepackage{sectsty}
\usepackage{fontspec}
\usepackage{titlesec}

%% Set page size with a wider right margin
\geometry{a4paper, total={170mm,257mm}, left=20mm, top=20mm, bottom=20mm, right=50mm}

%% Let's define some colours
\definecolor{light}{HTML}{E6E6FA}
\definecolor{highlight}{HTML}{800080}
\definecolor{dark}{HTML}{330033}

%% Let's add the border on the right hand side 
% \AddToShipoutPicture{% 
%     \AtPageLowerLeft{% 
%         \put(\LenToUnit{\dimexpr\paperwidth-3cm},0){% 
%             \color{light}\rule{3cm}{\LenToUnit\paperheight}%
%           }%
%      }%
%      % logo
%     \AtPageLowerLeft{% start the bar at the bottom right of the page
%         \put(\LenToUnit{\dimexpr\paperwidth-2.25cm},27.2cm){% move it to the top right
%             \color{light}\includegraphics[width=1.5cm]{_extensions/nrennie/PrettyPDF/logo.png}
%           }%
%      }%
% }

%% Style the page number
\fancypagestyle{mystyle}{
  \fancyhf{}
  \renewcommand\headrulewidth{0pt}
  \fancyfoot[R]{\thepage}
  \fancyfootoffset{3.5cm}
}
\setlength{\footskip}{20pt}

%% style the chapter/section fonts
\chapterfont{\color{dark}\fontsize{20}{16.8}\selectfont}
\sectionfont{\color{dark}\fontsize{20}{16.8}\selectfont}
\subsectionfont{\color{dark}\fontsize{14}{16.8}\selectfont}
\titleformat{\subsection}
  {\sffamily\Large\bfseries}{\thesection}{1em}{}[{\titlerule[0.8pt]}]
  
% left align title
\makeatletter
\renewcommand{\maketitle}{\bgroup\setlength{\parindent}{0pt}
\begin{flushleft}
  {\sffamily\huge\textbf{\MakeUppercase{\@title}}} \vspace{0.3cm} \newline
  {\Large {\@subtitle}} \newline
  \@author
\end{flushleft}\egroup
}
\makeatother

%% Use some custom fonts
\setsansfont{Ubuntu}[
    Path=_extensions/nrennie/PrettyPDF/Ubuntu/,
    Scale=0.9,
    Extension = .ttf,
    UprightFont=*-Regular,
    BoldFont=*-Bold,
    ItalicFont=*-Italic,
    ]

\setmainfont{Ubuntu}[
    Path=_extensions/nrennie/PrettyPDF/Ubuntu/,
    Scale=0.9,
    Extension = .ttf,
    UprightFont=*-Regular,
    BoldFont=*-Bold,
    ItalicFont=*-Italic,
    ]
\KOMAoption{captions}{tableheading}
\makeatletter
\@ifpackageloaded{caption}{}{\usepackage{caption}}
\AtBeginDocument{%
\ifdefined\contentsname
  \renewcommand*\contentsname{Table of contents}
\else
  \newcommand\contentsname{Table of contents}
\fi
\ifdefined\listfigurename
  \renewcommand*\listfigurename{List of Figures}
\else
  \newcommand\listfigurename{List of Figures}
\fi
\ifdefined\listtablename
  \renewcommand*\listtablename{List of Tables}
\else
  \newcommand\listtablename{List of Tables}
\fi
\ifdefined\figurename
  \renewcommand*\figurename{Figure}
\else
  \newcommand\figurename{Figure}
\fi
\ifdefined\tablename
  \renewcommand*\tablename{Table}
\else
  \newcommand\tablename{Table}
\fi
}
\@ifpackageloaded{float}{}{\usepackage{float}}
\floatstyle{ruled}
\@ifundefined{c@chapter}{\newfloat{codelisting}{h}{lop}}{\newfloat{codelisting}{h}{lop}[chapter]}
\floatname{codelisting}{Listing}
\newcommand*\listoflistings{\listof{codelisting}{List of Listings}}
\makeatother
\makeatletter
\makeatother
\makeatletter
\@ifpackageloaded{caption}{}{\usepackage{caption}}
\@ifpackageloaded{subcaption}{}{\usepackage{subcaption}}
\makeatother
\makeatletter
\@ifpackageloaded{tcolorbox}{}{\usepackage[skins,breakable]{tcolorbox}}
\makeatother
\makeatletter
\@ifundefined{shadecolor}{\definecolor{shadecolor}{rgb}{.97, .97, .97}}{}
\makeatother
\makeatletter
\@ifundefined{codebgcolor}{\definecolor{codebgcolor}{named}{light}}{}
\makeatother
\makeatletter
\ifdefined\Shaded\renewenvironment{Shaded}{\begin{tcolorbox}[boxrule=0pt, breakable, colback={codebgcolor}, enhanced, sharp corners, frame hidden]}{\end{tcolorbox}}\fi
\makeatother
\usepackage{bookmark}
\IfFileExists{xurl.sty}{\usepackage{xurl}}{} % add URL line breaks if available
\urlstyle{same}
\hypersetup{
  pdftitle={COMM 7018: Social Media Theory},
  colorlinks=true,
  linkcolor={highlight},
  filecolor={Maroon},
  citecolor={Blue},
  urlcolor={highlight},
  pdfcreator={LaTeX via pandoc}}


\title{COMM 7018: Social Media Theory}
\author{}
\date{}
\begin{document}
\maketitle

\pagestyle{mystyle}


\begin{figure}

\begin{minipage}{0.48\linewidth}
\href{http://polyducks.co.uk/}{\pandocbounded{\includegraphics[keepaspectratio]{img/polyducks.png}}}\end{minipage}%
%
\begin{minipage}{0.04\linewidth}
~\end{minipage}%
%
\begin{minipage}{0.48\linewidth}
\href{https://fitchburgstate.edu}{Fitchburg State University}\\
\href{https://www.fitchburgstate.edu/academics/academic-schools/school-arts-and-sciences/communications-media-department}{Communications
Media Department}\\
\href{https://www.fitchburgstate.edu/academics/programs/social-media-concentration-applied-communication-ms-online}{MS
in Applied Communication: Social Media Concentration}\\
GCE Online-Accelerated\\
7 weeks, Mon 19 May---Sun 6 July 2025\\
Instructor: Dr.~Martin Roberts\\
\href{https://mroberts1.github.io/fsu-smt-su24}{Syllabus}\\
\href{https://github.com/mroberts1/fsu-smt-su24}{GitHub repository}
\end{minipage}%

\end{figure}%

\subsection{Introduction}\label{introduction}

The term \textbf{social media} is popularly understood as referring to
corporate-owned, advertising-funded communication \textbf{platforms}
based on \textbf{user-generated content}: YouTube, Instagram, Facebook,
Twitter, Twitch, Discord, TikTok. It can also be defined more broadly,
however, as a set of networked, technologically-mediated
\textbf{practices} of communication, structured by economic and
political forces that both inflect and are inflected by social and
cultural identities. These platforms, the social practices that they
enable, and the relationship between the two are the objects of
\textbf{social media theory}. But what does it mean to theorize social
media? Why do we need social media theory at all?

To theorize something involves a number of processes

\begin{itemize}
\tightlist
\item
  first, how do we define the phenomenon or object of study itself? How
  does it differ from previous or other related phenomena?
\item
  how are we to account for it? Why did it happen/is it happening now
  rather than at some other time? What are its conditions of
  possibility?
\item
  what is its relation to larger areas of society? What are its
  implications for those areas?
\item
  how are we to evaluate it, in terms of its implications (political,
  economic, social, ethical, legal, environmental, aesthetic)? What are
  its possibilities and limits, its progressive and oppressive aspects?
  How can we change it for the better?
\end{itemize}

These processes involve developing analytical frameworks or models
comprising concepts that are useful for identifying and analyzing key
aspects of and issues raised by the phenomenon/object in question. These
frameworks and concepts typically draw from existing ones in different
fields of study, but often involve the proposal of new frameworks and
concepts specific to the field in question.

\subsection{Objectives}\label{objectives}

By the end of the course, students will be able to:

\begin{itemize}
\tightlist
\item
  analyze technologies past, present, and imagined
\item
  describe the ways in which technologies shape our world the ways in
  which we shape those technologies
\item
  explain how social media is a result of the intersection between
  technologies and existing human communication dynamics
\item
  discuss how theory of technology and social media can improve the
  vocational outlook of a student
\item
  play a productive role in and facilitate conversations that tease out
  the relationships between values and technology.\\
\item
  through the skills you will refine in writing your research papers,
  clearly explain how a specific technology shapes the social world that
  we live in.
\end{itemize}

\subsection{Reading Assignments}\label{reading-assignments}

All weekly reading assignments are linked directly to online sources.
Please download and print out PDFs during the first week of the course.

\subsection{Course Information}\label{course-information}

\textbf{Platforms}\\
We'll be using Blackboard for submitting assignments ONLY. For
discussion, we will be using
\href{https://discord.gg/GUSz99EP}{Discord}.

On Discord, if you don't already have an account, please set one up
using your Fitchburg State University email address as ID.

If you already have a Discord account there will be problems setting up
an account for the course because each account has to be tied to a
different phone number, so you will not be able to use your regular
phone number for verification. For this reason, you may use your
existing account, but if you use a pseudonym please let me know what
this is so I know who you are!

Please be sure to check in to the site at least once daily M-F to check
the Announcements page and the Discussion forum for the week.

\subsection{Assignments / Evaluation}\label{assignments-evaluation}

\begin{itemize}
\tightlist
\item
  \textbf{Review}: 6, weekly from Week 1, one short post responding to
  at least one of the readings, 250 words (maximum), due by Sunday
  (15\%)\\
\item
  \textbf{Discussion}: weekly after Week 1, 2-3 responses to other
  students' posts., due by the \emph{following} Sunday (15\%)\\
\item
  \textbf{Commentary}: 2 short papers, 1000 words, due Sunday of Week 3
  and Week 5 (20\%)\\
\item
  \textbf{Keywords}: 500-750 words on a key concept in social media
  theory, with bibliography/references, due Sunday of Week 4 (25\%)\\
\item
  \textbf{Platform Case Study}: 2,000 words, due Sunday of Week 7 (25\%)
\end{itemize}

\textbf{Discussion: Agenda, Review, Reply Posts}\\
For Weeks 1-6, each of the weekly topics will be active across a cycle
of two weeks.

By \textbf{Wednesday} of each week, I will post an Agenda item in the
Discussion forum for the topic of the week that discusses the reading
assignments for the week, identifying key themes, concepts, and/or
issues to look out for as you read. Be sure to read the Agenda post
before beginning the reading assignments.

In the first week, complete the reading assignments and make an initial
response post called a Review, with question and/or comments on them, by
\textbf{Sunday} of the week in question.

In the second week, read through the Review posts of the group and post
at least one Reply to one of them by Friday of that week.

\textbf{Commentary Papers}\\
These short papers (750-1,000 words) are due at the end of Week 3 and
Week 5 (Sunday). They should consist of close analytical readings of any
of the reading assignments for the period Weeks 1-3 or 4-5. You are
encouraged to focus in detail on particular sections, arguments, and/or
concepts from the readings and develop them.

\textbf{Platform Case Study}\\
The culminating written assignment for the course (2,000 words) may
consist of a research paper or report.

A 1-page preliminary proposal with ideas for your project, with a short
bibliography with sources and/or links, should be posted in the
Discussion forum for the purpose by the end of Week 5, and you will
receive feedback during Week 6.

\subsection{Journals and other research
resources}\label{journals-and-other-research-resources}

A very useful reference guide for research:

Anabel Quan-Haase and Luke Sloan, eds., \textbf{The SAGE Handbook of
Social Media Research Methods}. Second edition. London: SAGE Books,
2022.

Includes large sections on both qualitative and quantitative reserach
methods, as well as chapters specifically on doing research on YouTube,
TikTok, and other platforms.

You are recommended to add all of the sources below to your bookmarks
bar for easy access.

\href{https://www.creator-studies.com/}{\textbf{Creators and Platform
Labor Working Group}} (Cornell University)\\
\href{https://datasociety.net/}{\textbf{Data \& Society}}\\
\href{https://www.e-flux.com/}{\textbf{e-flux Journal}}\\
\href{https://www.invisibleculturejournal.com/}{\textbf{Invisible
Culture: A Journal for Visual Culture}}\\
\href{https://journals.sagepub.com/home/NMS}{\textbf{New Media \&
Society}}\\
\href{https://www.newmodels.io/}{\textbf{New Models}}\\
\href{https://journals.sagepub.com/home/SMS}{\textbf{Social Media +
Society}}\\
\href{https://tiktokcultures.com/}{\textbf{TikTok Cultures Research
Network}}\\
\href{https://reallifemag.com/}{\textbf{Real Life}} (now discontinued
online magazine about technology and everyday life)

\subsection{Keywords}\label{keywords}

\begin{itemize}
\tightlist
\item
  Algospeak
\item
  Below the radar
\item
  Digital ethnography
\item
  Imagined audience
\item
  Imitation publics
\item
  Livestreaming
\item
  Parasociality
\item
  Platformization
\end{itemize}

\subsection{Research Topics}\label{research-topics}

\begin{itemize}
\tightlist
\item
  Film TikTok
\item
  \#Booktok
\item
  Global Labor and the Gig Economy
\item
  Dark Forest Theory of the Internet
\end{itemize}

\subsection{Schedule}\label{schedule}

\emph{Week 1} M 05/19

\textbf{I. Theorizing Social Media}

\textbf{Infinite Content}

\begin{itemize}
\tightlist
\item
  \href{https://carolinebusta.github.io/}{Caroline Busta},
  ``\href{https://www.documentjournal.com/2024/05/technical-images-film01-angelicism-art-showtime-true-detective-shein/}{Hallucinating
  sense in the era of infinity-content}''\footnote{This text explores
    how the internet and media are changing how we assign meaning to
    content, moving beyond traditional text-based communication. It
    suggests that in a time of infinite content, our ability to sense
    and interpret non-linear information is becoming essential. The
    shift towards sensory-based communication may impact how we perceive
    reality and interact with technology. ---GPT3.5}
  (\href{https://www.documentjournal.com/}{\textbf{Document}}, 29 May
  2024)
  {[}\href{https://soundcloud.com/newmodels/nm-reads-hallucinating-sense-in-the-era-of-infinity-content-by-caroline-busta}{Audio}{]}
\item
  Nicole Lipman,
  ``\href{https://www.nplusonemag.com/issue-47/reviews/super-cute-please-like}{Super
  Cute Please Like}'' (\textbf{n+1} 47 (Spring 2024))\footnote{SHEIN is
    a popular fast-fashion brand known for its wide range of trendy and
    affordable clothing. The brand quickly produces garments,
    capitalizing on micro trends and offering low prices. Despite
    criticisms of exploitation and environmental impact, SHEIN's unique
    marketing strategies and expansive catalog make it a prominent
    player in the fashion industry. ---GPT3.5}
\item
  Timo Kollbrunner,
  ``\href{https://stories.publiceye.ch/en/shein/}{Toiling Away for
  Shein}'' (\href{https://www.publiceye.ch/en/}{\textbf{Public Eye}},
  November 2021)
\end{itemize}

\begin{center}\rule{0.5\linewidth}{0.5pt}\end{center}

\emph{Week 2} M 05/26

\textbf{Our Algorithms, Ourselves}

\begin{itemize}
\tightlist
\item
  \href{https://sophiebishop.co.uk/}{Sophie Bishop} and Tanya Kant,
  ``\href{https://journals.sagepub.com/doi/epub/10.1177/00380261221146403}{Algorithmic
  autobiographies and fictions: A digital method}''
\item
  \href{https://sophiebishop.co.uk/}{Sophie Bishop},
  ``\href{https://journals.sagepub.com/doi/epub/10.1177/2056305119897323}{Algorithmic
  Experts: Selling Algorithmic Lore on YouTube}''
\item
  \href{https://www.kylechayka.com/}{Kyle Chayka},
  ``\href{pdf/filterworld-intro.pdf}{Introduction}''
  (\textbf{Filterworld})
\item
  \href{https://www.kylechayka.com/}{Kyle Chayka},
  ``\href{pdf/filterworld-ch1.pdf}{The Rise of Algorithmic
  Recommendations}'' (\textbf{Filterworld}, ch.~1)
\item
  See also: Taylor Lorenz,
  ``\href{https://www.washingtonpost.com/technology/2022/04/08/algospeak-tiktok-le-dollar-bean/}{Internet
  `algospeak' is changing our language in real time, from `nip nops' to
  `le dollar bean'}'' (\textbf{Washington Post}, 8 April 2022)
  {[}\href{pdf/algospeak.pdf}{pdf}{]}
\end{itemize}

\begin{center}\rule{0.5\linewidth}{0.5pt}\end{center}

\emph{Week 3} M 06/02

\textbf{Influencers}

\begin{itemize}
\tightlist
\item
  \href{https://sophiebishop.co.uk/}{Sophie Bishop},
  ``\href{https://reallifemag.com/influencer-creep/}{Influencer Creep}''
\item
  \href{https://sophiebishop.co.uk/}{Sophie Bishop},
  ``\href{https://journals.sagepub.com/doi/epub/10.1177/14614448231206090}{Influencer
  creep: How artists strategically navigate the platformisation of art
  worlds}''
\end{itemize}

See also:

\begin{itemize}
\tightlist
\item
  \href{https://sophiebishop.co.uk/}{Sophie Bishop},
  ``\href{https://sophiebishop.co.uk/wp-content/uploads/2023/12/how-to-research-online-influencers.pdf}{How
  to Research Online Influencers}''
\item
  \href{https://sophiebishop.co.uk/}{Sophie Bishop},
  ``\href{https://journals.sagepub.com/doi/epub/10.1177/20563051211003066}{Influencer
  Management Tools: Algorithmic Cultures, Brand Safety, and Bias}''
\end{itemize}

\begin{center}\rule{0.5\linewidth}{0.5pt}\end{center}

\emph{Week 4} M 06/09

\textbf{II. Platform Cultures}

\textbf{Books}

\begin{itemize}
\tightlist
\item
  Jessica Maddox and Fiona Gill,
  ``\href{https://journals.sagepub.com/doi/epub/10.1177/20563051231213565}{Assembling
  `Sides' of TikTok: Examining Community, Culture, and Interface through
  a BookTok Case Study}''
\item
  José M. Tomasena,
  ``\href{https://journals.sagepub.com/doi/epub/10.1177/2056305119894004}{Negotiating
  Collaborations: BookTubers, The Publishing Industry, and YouTube's
  Ecosystem}''
\end{itemize}

See also:

\begin{itemize}
\tightlist
\item
  Alysia De Melo,
  ``\href{https://journals.sagepub.com/doi/epub/10.1177/20563051241286700}{The
  Influence of BookTok on Literary Criticisms and Diversity}''
\item
  Michael Dezuanni and Amy Schoonens,
  ``\href{https://journals.sagepub.com/doi/epub/10.1177/20563051241309499}{\#BookTok's
  Peer Pedagogies: Invitations to Learn About Books and Reading on
  TikTok}''
\end{itemize}

\begin{center}\rule{0.5\linewidth}{0.5pt}\end{center}

Break (due to instructor bereavement)

\begin{center}\rule{0.5\linewidth}{0.5pt}\end{center}

\emph{Week 5} M 06/23

\textbf{Music}

\begin{itemize}
\tightlist
\item
  David Hesmondhalgh, Ellis Jones, and Andreas Rauh,
  ``\href{https://journals.sagepub.com/doi/epub/10.1177/2056305119883429}{Soundcloud
  and Bandcamp as Alternative Music Platforms}''
\item
  Jeremy Wade Morris,
  ``\href{https://journals.sagepub.com/doi/epub/10.1177/2056305120940690}{Music
  Platforms and the Optimization of Culture}''
\item
  Robin James, ``\href{https://reallifemag.com/moving-in-stereo/}{Moving
  in Stereo}''
\item
  D. Bondy Valdovinos Kaye,
  ``\href{https://scholarlypublishingcollective.org/uip/jac/article/6/2/92/383572/JazzTok-Creativity-Community-and-Improvisation-on}{JazzTok:
  Creativity, Community, and Improvisation on TikTok}''
\end{itemize}

\begin{center}\rule{0.5\linewidth}{0.5pt}\end{center}

\emph{Week 6} M 06/30

\textbf{Media}

\begin{itemize}
\tightlist
\item
  \href{https://www.sunsunlim.com/}{Sun Sun Lim},
  ``\href{https://journals.sagepub.com/doi/epub/10.1177/2056305115578137}{On
  Stickers and Communicative Fluidity in Social Media}''
\item
  Sunny Yoon,
  ``\href{https://journals.sagepub.com/doi/epub/10.1177/20563051241292577}{Webtoons,
  Desperately Seeking Viewers: Interactive Creativity in Social Media
  Platforms and Cultural Appropriation of Global Media Production}''
\item
  Kate M. Miltner and Tim Highfield,
  ``\href{https://journals.sagepub.com/doi/epub/10.1177/2056305117725223}{Never
  Gonna GIF You Up: Analyzing the Cultural Significance of the Animated
  GIF}''
\item
  Luke Stark and Kate Crawford,
  ``\href{https://journals.sagepub.com/doi/epub/10.1177/2056305115604853}{The
  Conservatism of Emoji: Work, Affect, and Communication}''
\end{itemize}

\begin{center}\rule{0.5\linewidth}{0.5pt}\end{center}

\emph{Week 7} M 07/07

\textbf{Aesthetics}

\begin{itemize}
\tightlist
\item
  Robin James, ``\href{https://reallifemag.com/new-normal/}{New
  Normal}''
\item
  Paul Roquet, ``\href{https://reallifemag.com/in-the-mood/}{In the
  Mood}''
\item
  Guilherme Giolo and Michaël Berghman, ``The Aesthetics of the Self:
  The Meaning-Making of Internet Aesthetics''
\item
  See also:
  \href{https://aesthetics.fandom.com/wiki/Aesthetics_Wiki}{Aesthetics
  Wiki} (read all articles in the section called ``What Are
  Aesthetics?'' and explore the site)
\end{itemize}

\begin{center}\rule{0.5\linewidth}{0.5pt}\end{center}

\subsection{Resources}\label{resources}

\href{https://sophiebishop.co.uk/}{Sophie Bishop}~(works on influencer
culture)\\
\href{https://www.abbiesr.com/about}{Abbie Richards} (TikTok researcher,
Media Matters)

\subsection{Bibliography}\label{bibliography}

D. Bondy Valdovinos Kaye, Jing Zeng, and Patrik Wikström,
\textbf{TikTok: Creativity and culture in short video}. Cambridge:
Polity Press, 2022.

danah boyd, \textbf{It's Complicated: The Social Lives of Networked
Teens} (New Haven: Yale University Press, 2014).

Amy Bruckman, \textbf{Should You Believe Wikipedia? Online Communities
and the Construction of Knowledge} (Cambridge: Cambridge University
Press, 2022).

Finn Brunton and Helen Nissenbaum, \textbf{Obfuscation: A User's Guide
for Privacy and Protest} (Cambridge: MIT Press, 2016).

Alessandro Caliandro and James Graham,
``\href{https://journals.sagepub.com/doi/epub/10.1177/2056305120924779}{Studying
Instagram Beyond Selfies}''

\href{https://www.kylechayka.com/}{Kyle Chayka}, \textbf{Filterworld:
How Algorithms Flattened Culture}. New York: Doubleday, 2024.

Gabriella Coleman, \textbf{Hacker, Hoaxer, Whistleblower, Spy: The Many
Faces of Anonymous} (London and New York: Verso, 2014).

Claire Dederer, \textbf{Monsters: A Fan's Dilemma} (New York: Alfred A.
Knopf, 2023).

\href{https://emerson.edu/faculty-staff-directory/kate-eichhorn}{Kate
Eichhorn},
\href{https://mitpress.mit.edu/9780262543286/content/}{\textbf{Content}}.
Essential Knowledge Series. Cambridge, MA: MIT Press, 2022.

Alysia De Melo,
``\href{https://journals.sagepub.com/doi/epub/10.1177/20563051241286700}{The
Influence of BookTok on Literary Criticisms and Diversity}''

Michael Dezuanni and Amy Schoonens,
``\href{https://journals.sagepub.com/doi/epub/10.1177/20563051241309499}{\#BookTok's
Peer Pedagogies: Invitations to Learn About Books and Reading on
TikTok}''

Brooke Erin Duffy, Thomas Poell, and David B Nieborg,
``\href{https://journals.sagepub.com/doi/epub/10.1177/2056305119879672}{Platform
Practices in the Cultural Industries: Creativity, Labor, and
Citizenship}''

Larissa Hjorth and Natalie Hendry,
``\href{https://journals.sagepub.com/doi/epub/10.1177/2056305115580478}{A
Snapshot of Social Media: Camera Phone Practices}''

Sarah J. Jackson, Moya Bailey, et al., \textbf{\#Hashtag Activism:
Networks of Race and Gender Justice} (Cambridge: MIT Press, 2020).

D. Bondy Valdovinos Kaye,
``\href{https://watermark.silverchair.com/92kaye.pdf?token=AQECAHi208BE49Ooan9kkhW_Ercy7Dm3ZL_9Cf3qfKAc485ysgAAA0IwggM-BgkqhkiG9w0BBwagggMvMIIDKwIBADCCAyQGCSqGSIb3DQEHATAeBglghkgBZQMEAS4wEQQMUQ3G4vO_yq2mYcrPAgEQgIIC9cFNcqJ-tvkfDkABZcgdqxRKk2nBHZwwo0JsNBv506DO0UmmiXXZMCVhjbGJulFTAD9GNWkNkMkdGiceXop6meQPzGcowi2352fEas53GVIEP6OfBSz7nGE3GDB5HeOgxIaNlYoBpu4eOkZuT9ylyL9nyJnLW728HsU6PoXyAzmvO21RZthx1-VDaii8db72kmS9SX754tbugG5z97yra1Vr7ZDZHcM6lWv_ot5YjNfgHX_YI0LUTbgorUsAS374e6ovCundjdohOPc0EzGKKsH3oB1TJ8t7KKM6-OUlGIhkNfHUZW27SwEHe9cysKfwKnIh2J3MdKs_Vb_5uq_glDkntrxoIaLL_H_ygoOhPpIUfeXhkaeCVTydh-Z81oOjo5T_d7wHLqmPGFR4ury1BQSv47KFmWcWNnamBqUOFqbe1sXZsjF4-LAjoTqERwmjA8KowI9oR8X6O0uFgfPL7X0lWa_WhIsTWiKtmHyOpteJeOmlSU63bf4KAPox_9SU0YahHIK1pd9hEfp7vY4h7zE88EYj9L8e_aTa37rNtt165htvmcuVE8Ed_9qCZghryDIy93hjLbBUIQbQvC_t-kv2PCYQRAkx7HjSIGnTqSogR0H8oxUX00LgHdfjqQF6LZoju0qhFfKyEbzWniLxr-iHZ4ESt6jgve5KGYDQNLVe9FZmeGMEayInt4izRmhvdunONM2nkhdeIYlDAaIoyfqIDXUqupLP1V_V6vGMVVb1-IY6RjGsj-T8TL4fM4K8WQT9LtytSmm2LXbqTH7QF91OoYcN-8_0WtY1UGr7KbRdg5bqKQpYQw5os3TpXOiNgS81QkP25iy-bQS4HU_d6jtubjXgcYYeHR_tNVTyY3ldUMiEKoJwAOKShlIzlAsc1LC0-e3VurvTPnyRkvZiSclNDceco2aUwR6Ld6WxexZvyltprjN5wxYBIx2k9EfnmXXiOYyMOswMU4ICOZEaw4Ba6qg7rfCTO4P6P1P3hSjS5jBakZY}{JazzTok:
Creativity, Community, and Improvisation on TikTok},'' \textbf{Jazz and
Culture} 6 (2) (2023): 92--116.

\href{https://x.com/lagorio}{Christine Lagorio-Chafkin}, \textbf{We Are
the Nerds: The Birth and Tumultuous Life of Reddit, the Internet's
Culture Laboratory}. New York: Hachette Books, 2018.

Jin Lee and Crystal Abidin,
``\href{https://journals.sagepub.com/doi/epub/10.1177/20563051231157452}{Introduction
to the Special Issue of `TikTok and Social Movements'}''

Jessica Maddox and Fiona Gill,
``\href{https://journals.sagepub.com/doi/epub/10.1177/20563051231213565}{Assembling
`Sides' of TikTok: Examining Community, Culture, and Interface through a
BookTok Case Study}''

Gary Marcus \& Ernest Davis, \textbf{Rebooting AI: Building Artificial
Intelligence We Can Trust} (New York: Pantheon Books, 2019).

Angela Nagle, \textbf{Kill All Normies: Online Culture Wars From 4Chan
and Tumblr to Trump and the Alt-Right} (Alresford, Hampshire, UK: Zero
Books, 2017). * Cathy O'Neil, with Stephen Baker, \textbf{The Shame
Machine: Who Profits in the New Age of Humiliation} (New York:
Crown/Random House, 2022).

David B Nieborg, Brooke Erin Duffy, and Thomas Poell,
``\href{https://journals.sagepub.com/doi/epub/10.1177/2056305120943273}{Studying
Platforms and Cultural Production: Methods, Institutions, and
Practices}''

\href{https://x.com/parmy}{Parmy Olson}, \textbf{Supremacy: AI, ChatGPT,
and the Race that Will Change the World}. New York: MacMillan, 2024.

Whitney Phillips, \textbf{This Is Why We Can't Have Nice Things: Mapping
the Relationship between Online Trolling and Mainstream Culture}
(Cambridge: MIT Press, 2015).

Whitney Phillips and Ryan M. Milner, \textbf{You Are Here: A Field Guide
for Navigating Polarized Speech, Conspiracy Theories, and Our Polluted
Media Landscape} (Cambridge: MIT Press, 2021).

Nicholas Proferes et al.,
``\href{https://journals.sagepub.com/doi/epub/10.1177/20563051211019004}{Studying
Reddit: A Systematic Overview of Disciplines, Approaches, Methods, and
Ethics}''

\href{https://allissavrichardson.com}{Allissa V. Richardson},
\textbf{Bearing Witness While Black: African Americans, Smartphones, and
the New Protest \#Journalism}. Oxford: Oxford University Press, 2020.

Jonathan Schroeder,
``\href{https://www.invisibleculturejournal.com/pub/snapshot-aesthetics/release/1}{Snapshot
Aesthetics and the Strategic Imagination}''
(\href{https://www.invisibleculturejournal.com/}{\textbf{Invisible
Culture: A Journal for Visual Culture}}) 10 (10 April 2013)

Kseniya Stsiampkouskaya et al.,
``\href{https://journals.sagepub.com/doi/epub/10.1177/20563051211035692}{Imagined
Audiences, Emotions, and Feedback Expectations in Social Media Photo
Sharing}''

José M. Tomasena,
``\href{https://journals.sagepub.com/doi/epub/10.1177/2056305119894004}{Negotiating
Collaborations: BookTubers, The Publishing Industry, and YouTube's
Ecosystem}''

\subsection{Policies}\label{policies}

\textbf{Late Policy}

Assignments that are late will lose 1/2 of a grade per day, beginning at
the end of class and including weekends and holidays. This means that a
paper, which would have received an A if it was on time, will receive a
B+ the next day, B- for two days late, and so on. Time management,
preparation for our meetings, and timely submission of your work
comprise a significant dimension of your professionalism. As such, your
work must be completed by the beginning of class on the day it is due.
If you have a serious problem that makes punctual submission impossible,
you must discuss this matter with me before the due date so that we can
make alternative arrangements. Because you are given plenty of time to
complete your work, and major due dates are given to you well in
advance, last minute problems should not preclude handing in assignments
on time.

\begin{center}\rule{0.5\linewidth}{0.5pt}\end{center}

\textbf{Mandatory Reporter}

Fitchburg State University is committed to providing a safe learning
environment for all students that is free of all forms of discrimination
and harassment. Please be aware all FSU faculty members are ``mandatory
reporters,'' which means that if you tell me about a situation involving
sexual harassment, sexual assault, dating violence, domestic violence,
or stalking, I am legally required to share that information with the
Title IX Coordinator. If you or someone you know has been impacted by
sexual harassment, sexual assault, dating or domestic violence, or
stalking, FSU has staff members trained to support you. If you or
someone you know has been impacted by sexual harassment, sexual assault,
dating or domestic violence, or stalking, please visit
\url{http://fitchburgstate.edu/titleix} to access information about
university support and resources.

\begin{center}\rule{0.5\linewidth}{0.5pt}\end{center}

\textbf{Health}

\href{http://www.google.com/url?q=http\%3A\%2F\%2Fwww.fitchburgstate.edu\%2Foffices-services-directory\%2Fhealth-services\%2F&sa=D&sntz=1&usg=AFQjCNEw5V0i0hL5DVO5b43gejNNaAt4ig}{Health
Services}

Hours: Monday-Friday 8:30AM-5PM Location: Ground Level of Russell Towers
(across from the entrance of Holmes Dining Hall) Phone: (978)
665-3643/3894

\href{http://www.google.com/url?q=http\%3A\%2F\%2Fwww.fitchburgstate.edu\%2Foffices-services-directory\%2Fcounseling-services\%2F&sa=D&sntz=1&usg=AFQjCNEYiS4EmSvWerpp2bKr5lTpouPuqQ}{Counseling
Services}

The Counseling Services Office offers a range of services including
individual, couples and group counseling, crisis intervention,
psychoeducational programming, outreach ALTERNATIVE ECOSYSTEMSs, and
community referrals. Counseling services are confidential and are
offered at no charge to all enrolled students. Staff at Counseling
Services are also available for consultation to faculty, staff and
students. Counseling Services is located in the Hammond, 3rd Floor, Room
317.

\href{http://www.google.com/url?q=http\%3A\%2F\%2Fwww.fitchburgstate.edu\%2Foffices-services-directory\%2Ffitchburg-anti-violence-education\%2F&sa=D&sntz=1&usg=AFQjCNFi5qy-wunMxX-hoWbA9YwT8aa4Ig}{Fitchburg
Anti-Violence Education (FAVE)}

FAVE collaborates with a number of community partners (e.g., YWCA
Domestic Violence Services, Pathways for Change) to meet our training
needs and to link survivors with community based resources. This site
also features
\href{http://www.google.com/url?q=http\%3A\%2F\%2Fwww.fitchburgstate.edu\%2Foffices-services-directory\%2Ffitchburg-anti-violence-education\%2Ffitchburg-anti-violence-education-resources\%2F&sa=D&sntz=1&usg=AFQjCNF9KZ2O1AvPMLJTHdNg1DfmYYtgog}{resources}
for help or information about dating violence, domestic violence, sexual
assault and stalking. If you or someone you know is in an abusive
relationship or has been a victim of sexual assault, there are many
places to go for help. Many can be accessed 24 hours a day, seven days a
week, 365 days a year. On campus, free and confidential support is
provided at both Counseling Services and Health Services.

\emph{Community Food Pantry} Food insecurity is a growing issue and it
certainly can affect student learning. The ability to have access to
nutritious food is incredibly vital. The Falcon Bazaar, located in
Hammond G 15, is stocked with food, basic necessities, and can provide
meal swipes to support all Fitchburg State students experiencing food
insecurity for a day or a semester.

The university continues to partner with Our Father's House to support
student needs and access to food and services. All Fitchburg State
University students are welcome at the Our Father's House Community Food
Pantry. This Pantry is located at the Faith Christian Church at 40
Boutelle St., Fitchburg, MA and is open from 5-7pm. Each ``household''
may shop for nutritious food once per month by presenting a valid FSU
ID.

\begin{center}\rule{0.5\linewidth}{0.5pt}\end{center}

\textbf{Academic Integrity}

The University ``Academic Integrity'' policy can be found online at
\href{http://www.fitchburgstate.edu/offices-services-directory/office-of-student-conduct-mediation-education/academic-integrity/}{http://
www.fitchburgstate.edu/offices-services-directory/office-of-student-conductmediation-education/academic-integrity/.}
Students are expected to do their own work. Plagiarism and cheating are
inexcusable. Any instance of plagiarism or cheating will automatically
result in a zero on the assignment and may be reported the Office of
Student and Academic Life at the discretion of the instructor.

Plagiarism includes, but is not limited to: - Using papers or work from
another class. - Using another student's paper or work from any class. -
Copying work or a paper from the Internet. - The egregious lack of
citing sources or documenting research.

\emph{If you're not clear on what is or is not plagiarism, ASK. The BEST
case scenario if caught is a zero on that assignment, and ignorance of
what does or does not count is not an excuse. That being said, I'm a
strong supporter of}
\href{https://en.wikipedia.org/wiki/Fair_Use}{\emph{Fair Use}}
\emph{doctrine. Just attribute what you use--and, again, ASK if there's
any doubt.}

\textbf{Americans With Disabilities Act (ADA)}

\begin{center}\rule{0.5\linewidth}{0.5pt}\end{center}

If you need course adaptations or accommodations because of a
disability, if you have emergency medical information to share with the
instructor, or if you need special arrangements in case the building
must be evacuated, please inform the faculty member as soon as possible.

\begin{center}\rule{0.5\linewidth}{0.5pt}\end{center}

\textbf{Technology}

At some point during the semester you will likely have a problem with
technology. Your laptop will crash; your iPad battery will die; a
recording you make will disappear; you will accidentally delete a file;
the wireless will go down at a crucial time. These, however, are
inevitabilities of life, not emergences. Technology problems are not
excuses for unfinished or late work. Bad things may happen, but you can
protect yourself by doing the following:

\begin{itemize}
\item
  Plan ahead: A deadline is the last minute to turn in material. You can
  start---and finish---early, particularly if challenging resources are
  required, or you know it will be time consuming to finish this
  project.
\item
  Save work early and often: Think how much work you do in 10 minutes. I
  auto save every 2 minutes.
\item
  Make regular backups of files in a different location: Between Box,
  Google Drive, Dropbox, and iCloud, you have ample places to store and
  backup your materials. Use them.
\item
  Save drafts: When editing, set aside the original and work with a
  copy.
\item
  Practice safe computing: On your personal devices, install and use
  software to control viruses and malware.
\end{itemize}

\begin{center}\rule{0.5\linewidth}{0.5pt}\end{center}

\textbf{Grading}

Grading for the course will follow the FSU grading policy below:

4.0: 95-100\\
3.7: 92-94\\
3.5: 89-91\\
3.3: 86-88\\
3.0: 83-85\\
2.7: 80-82\\
2.5: 77-79\\
2.3: 74-76\\
2.0: 71-73\\
0.0: \textless{} 70

\begin{center}\rule{0.5\linewidth}{0.5pt}\end{center}

\textbf{Academic Resources}

\href{http://www.fitchburgstate.edu/offices-services-directory/tutor-center/writing-help/}{Writing
Center}

\href{http://catalog.fitchburgstate.edu/content.php?catoid=13&navoid=851}{Academic
Policies}

\href{http://www.fitchburgstate.edu/offices-services-directory/disability-services/}{Disability
Services}

\href{https://www.getrave.com/login/fitchburgstate/}{Fitchburg State
Alert system for emergencies, snow closures/delays, and faculty
absences}

\href{http://www.fitchburgstate.edu/offices-services-directory/career-counseling-and-advising/careerservices/}{University
Career Services}

\begin{center}\rule{0.5\linewidth}{0.5pt}\end{center}




\end{document}
